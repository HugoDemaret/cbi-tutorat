\documentclass{article}
\usepackage[utf8]{inputenc}
\usepackage{algpseudocode}
\usepackage[]{algorithm2e}
\usepackage{tikz}
\usepackage{amsmath}
\usepackage{amssymb}
\usetikzlibrary{automata}

\title{Tutorat CBI : Module 1 (TP)}
\author{Hugo Demaret}
\date{2021-2022}
\begin{document}
\maketitle
    \section*{Partie I}
        \subsection*{Installation de Linux}
            \subsubsection*{Prérequis}
                \begin{table}[h]
                    \begin{tabular}{|l|c|c|p{5cm}}
                        \hline
                        \textit{Prérequis}&\textit{Cochez si acquis}\\
                        \hline
                        \hline
                        \textrm{Connaissance basique d'Unix}&\\
                        \hline
                        \textrm{Ordinateur}&\\
                        \hline
                        \textrm{Media d'installation (clef usb, carte SD...)}&\\
                        \hline
                    \end{tabular}
                \end{table}
            \subsubsection*{Création du media}
                \textit{Suivez les étapes ci-dessous :}\\
                \begin{algorithm}[H]
                    \KwData{Un media sans Linux, Liste des prérequis}
                    \KwResult{Un media avec Linux}
                    \While{(pas tout lu) AND (pas tout compris)}{
                        Lire jusqu'à comprendre\;
                        Poser des questions\;
                    }
                    \While{Media sans Linux}{
                        Télécharger l'image (en .iso) de la distribution de votre choix\;
                        \uIf{Sur Windows}{
                            Installer le logiciel "Rufus"\;
                            Lire la documentation de "Rufus"\;
                            Suivre la documentation de "Rufus"\;
                        }
                        \uElseIf{Sur MacOS}{
                            Installer le logiciel "Balena Etcher"\;
                            Suivre la documentation de "Balena Etcher"\;
                        }
                        \uElseIf{Sur Linux}{
                            Brancher le media\;
                            Touver le chemin d'accès au media\;
                            Démonter le media de l'ordinateur\;
                            Formater le media (exfat)\;
                            Utiliser la commande dd (attention, commande dangereuse)\;
                        }
                    }
                    \caption{Comment créer le média d'installation}
                   \end{algorithm}
        \subsection*{}
\end{document}
